\chapter{序論}

ソロモン諸島の\ruby[g]{共通言語}{リンガ・フランカ}であるPijin語は、名前の通りピジン・クレオール言語の1つであり、パプア・ニューギニアのトク・ピシン、バヌアツのビスラマ語と共にメラネシア・ピジンというグループを構成する。
これらの中で、Pijin語は最も英語の影響を受けていると広く認められている一方で、研究は3つの中で最も進んでいないとされる\footnote{
  Jourdanは次のように書いている:「ソロモン諸島のPijin語(SIP)---太平洋のピジンの中で最も研究されていない---は、最も英語化されたメラネシア・ピジンであるということは、太平洋のピジン・クレオール語学者たちの間にでさえ、広く伝わっている考えであるように思われる」
  ``It seems a widespread impression, even among Pacific pidgin and creole scholars, that Solomon Islands Pijin (SIP)---the least studied of the Pacific pidgins---is the most anglicized Melanesian pidgin.''\citep{nativization}
}\citep{nativization}。

事実、Pijin語についての研究資料として、語彙の面では非常に詳しく記された\cite{dictionary}があるが、包括的な文法書は未だに発行されておらず\citep{phonology}、文法は\cite{yumi}や\cite{eric}などの語学的な入門書から部分的に学ぶことしかできない。
トク・ピシンやビスラマ語がそれぞれの国で公用語の地位を与えられているのに対して、ソロモン諸島の公用語は未だに英語のみであるという政治的な地位の低さも、研究の少なさに拍車をかけている要因の1つといえるだろう。

本研究の目的は、近縁のピジン言語と比較しても未解明の部分が多く、近年の都市化や英語との言語接触によって激しい影響を受けていると言われる現代のPijin語文法を分析することであり、特にこの論文では、\textit{dat}と\textit{fo}という2つの単語の従属節標識としての用法の共時的な記述を目標とする。これら2つの単語は、既存の研究や語学資料には、本論で取り扱う従属節標識としての用法は示されていても非常に簡潔であるか、あるいは全く示されていない。それゆえ、これら2単語の従属節標識としての役割を示すことには一定の意味があるだろう。

本論では、まずはじめに本章の残りでPijin語の社会的状況や歴史的背景に対する簡単な説明をした上で、本研究で行ったPijin語聖書のコーパス分析や現地での調査について、その詳細を述べる。その後、第2章では\textit{dat}について、第3章では\textit{fo}について単語ごとにその分析や調査の結果を述べる。第4章では、第3章までの結果を振り返り、Pijin語の文法研究全体についての示唆を述べるつもりである。

%筆者は未解明な部分の多く、比較的近年に発行されたPijin語新約聖書のコーパス分析を行った。その結果、本来は別の意味を持っている2つの単語---\textit{dat}、\textit{fo}---が、これまでの研究や辞書にはほとんど記されていない従属節標識としての用法を持っていることが明らかになった。その後、ソロモン諸島の首都であるホニアラに2週間滞在する機会を得たため、これらの用法や容認度について現地での調査を行った。
\section{Pijin語について}

Pijin語はメラネシア・ピジンの1つである。

\section{調査の方法}

聖書のコーパス分析とホニアラでの現地調査を行った。
