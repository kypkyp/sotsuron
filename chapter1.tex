\chapter{序論}

ソロモン諸島の\ruby[g]{共通言語}{リンガ・フランカ}であるPijin語
\footnote{ソロモン諸島で話されているこの言語は、現地では単に\textit{pijin}あるいは\textit{tok pijin}と呼ばれる。英語の文献では種類としてのピジン言語を``pidgin language(s)''、この個別言語を``Pijin''あるいは``Solomon Islands Pijin(SIP)''と表記することが多いようである。本論においては、前者を「ピジン言語」、後者を「Pijin語」と呼んで区別することにする。}
は、名前の通りピジン・クレオール言語の1つであり、パプア・ニューギニアのトク・ピシン、バヌアツのビスラマ語と共にメラネシア・ピジンというグループを構成する。
これらの中で、Pijin語は最も英語の影響を受けていると広く認められている一方、研究は3つの中で最も進んでいないとされる\footnote{
  Jourdanは次のように書いている:「ソロモン諸島のPijin語(SIP)---太平洋のピジンの中で最も研究されていない---は、最も英語化されたメラネシア・ピジンであるということは、太平洋のピジン・クレオール語学者たちの間にでさえ、広く伝わっている考えであるように思われる」
  ``It seems a widespread impression, even among Pacific pidgin and creole scholars, that Solomon Islands Pijin (SIP)---the least studied of the Pacific pidgins---is the most anglicized Melanesian pidgin.''\citep{nativization}
}\citep{nativization}。

事実、Pijin語についての研究資料として、語彙の面では非常に詳しく記された\cite{dictionary}があるが、包括的な文法書は未だに発行されておらず\citep{phonology}、文法は\cite{yumi}や\cite{eric}などの語学的な入門書から部分的に学ぶことしかできない。
トク・ピシンやビスラマ語がそれぞれの国で公用語の地位を与えられているのに対して、ソロモン諸島の公用語は未だに英語のみであるという政治的な地位の低さも、研究の少なさに拍車をかけている要因の1つといえるだろう。

本研究の目的は、近縁のピジン言語と比較しても未解明の部分が多く、近年の都市化や英語との言語接触によって激しい影響を受けていると言われる現代のPijin語文法を分析することであり、特にこの論文では、\textit{dat}と\textit{fo}という2つの単語の従属節標識としての用法の共時的な記述を目標とする。これら2つの単語は、既存の研究や語学資料には、本論で取り扱う従属節標識としての用法は示されていても非常に簡潔であるか、あるいは全く示されていない。それゆえ、これら2単語の従属節標識としての役割を示すことには一定の意味があるだろう。

本論では、まずはじめに本章の残りでPijin語の社会的状況や歴史的背景に対する簡単な説明をした上で、本研究で行ったPijin語聖書のコーパス分析や現地での調査について、その詳細を述べる。その後、第2章では\textit{dat}について、第3章では\textit{fo}について単語ごとにその分析や調査の結果、そして簡単な考察を述べる。第4章では、第3章までの結果を振り返り、Pijin語の文法研究全体についての示唆を述べるつもりである。

%筆者は未解明な部分の多く、比較的近年に発行されたPijin語新約聖書のコーパス分析を行った。その結果、本来は別の意味を持っている2つの単語---\textit{dat}、\textit{fo}---が、これまでの研究や辞書にはほとんど記されていない従属節標識としての用法を持っていることが明らかになった。その後、ソロモン諸島の首都であるホニアラに2週間滞在する機会を得たため、これらの用法や容認度について現地での調査を行った。
\section{Pijin語について}

本論の目的はPijin語のあくまで共時的な研究であるが、文法やその変化に関連があると思われるPijin語の社会言語学的な状況を整理するために、本節では簡単にPijin語の歴史を解説した上で現在の言語状況についても述べる。\footnote{
ここに示されたPijin語の歴史については、特に断りがなければ\cite{phonology}に基づいている。その他、この節では取り上げていないものの、Pijin語を中心としてメラネシア・ピジンの歴史に迫った詳細な研究として\cite{keesing}があり、1850年代以前に中央太平洋地域(カロライン諸島、ロツマ島、フィジー諸島、ギルバート諸島)で話されていたいわゆるジャーゴン言語から、早期メラネシア・ピジンがソロモン諸島のPijin語へと変化していく過程でのソロモン諸島の現地語(特にマライタ島のKwaio語)の影響についてまで詳細な記述と検討を行っている。}

太平洋のピジン言語は、その起源を19世紀前半にまで遡ることができる\cite{keesing}が、現在メラネシア・ピジンと呼ばれる3言語---トク・ピシン、ビスラマ語、そしてPijin語---の原型がはっきりと形作られるのは19世紀後半のオーストラリアである。1863年から1906年までの43年間、オーストラリアのクイーンズランドで行われたプランテーションにはメラネシア全域から労働者が集められ、その中にはソロモン諸島からの13,000人の労働者も含まれていた。そのプランテーションで話されていたのが、まさに早期メラネシア・ピジンと呼ばれる英語を主な語彙供給源としたピジン言語であり、彼らは自身の契約期間、あるいはプランテーション自体の終了後にその言語をソロモン諸島全域に持ち帰ることになったのである。

1893年、ソロモン諸島がイギリスにより併合されると、このピジン言語は現地の島民が植民地政府のイギリス人職員や諸島内の他民族と話す手段としての社会的な役割を担うことになった。20世紀に入り、諸島内でもプランテーションが開始すると、この言語はさらに多くの話者を抱えることになる。太平洋戦争においては、ソロモン諸島はガダルカナル島を含む各地が激戦地となったが、この時期においてもピジン言語はコミュニケーションの媒体として大いに役立った。2,000人程度の現地人がソロモン諸島労働者部隊として、680人がイギリス領ソロモン諸島保護国防衛軍として米軍側で戦った記録が残っているが、アメリカ軍の兵士は、この地域に入る前に基礎的なレベルのトク・ピシンを学んだ記録があり、島民側の証言では兵士と話すのに大半はピジン語を、部分的には英語を用いたという証言を残している。第二次世界大戦後の政治運動(Maasina Ruru)でもPijin語はソロモン諸島内の異なる言語集団を結びつける大きな役割を果たし、独立後も都市化や学校教育\footnote{
一部の初等教育などの例外を除いて、学校で教科書や授業に用いられる言語は英語である一方で、より非公式な状況--登下校中や休憩時間の友達との会話---ではPijin語が用いられる。このため、学校に通う生徒は英語とPijin語の双方を覚えることが必須であり、さらにこれは後述する英語との激しいコードスイッチングの実例の1つといえるだろう。
}
の発展に伴って、Pijin語が使用される機会は劇的に増加していった。

ソロモン諸島は言語的に非常に多様であり、首都ホニアラでは64の土着語を話す集団がいるという\citep{nativization}。1つの土着語を話す集団はwantokと呼ばれ、島民は同じwantokの人の間ではその言語を、異なるwantokの間ではPijin語を用いるのがソロモン諸島の従来(あるいは非都市部)の言語習慣である。一方で、前述した通り多様な言語集団が併存し、wantokを跨いだ婚姻も増えているという首都ホニアラでは、1960年頃から日常的にPijin語が話されるようになり、さらに従来存在しなかったPijin語の母語話者が誕生するようになった。

%\cite{nativization}について述べる%
\section{調査の方法}\label{sec:howexamined}
