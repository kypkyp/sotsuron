\chapter{序論}

ソロモン諸島の\ruby[g]{共通言語}{リンガ・フランカ}であるPijin語
\footnote{ソロモン諸島で話されているこの言語は、現地では単に\textit{pijin}あるいは\textit{tok pijin}と呼ばれる。英語の文献では種類としてのピジン言語を``pidgin language(s)''、この個別言語を``Pijin''あるいは``Solomon Islands Pijin(SIP)''と表記することが多いようである。本論においては、前者を「ピジン言語」、後者を「Pijin語」と呼んで区別することにする。}
は、名前の通りピジン・クレオール言語の1つであり、パプア・ニューギニアのトク・ピシン、バヌアツのビスラマ語と共にメラネシア・ピジンというグループを構成する。
これらの中で、Pijin語は最も英語の影響を受けていると広く認められている一方、研究は3つの中で最も進んでいないとされる\footnote{
  Jourdanは次のように書いている:「ソロモン諸島のPijin語(SIP)---太平洋のピジンの中で最も研究されていない---は、最も英語化されたメラネシア・ピジンであるということは、太平洋のピジン・クレオール語学者たちの間にでさえ、広く伝わっている考えであるように思われる」
  ``It seems a widespread impression, even among Pacific pidgin and creole scholars, that Solomon Islands Pijin (SIP)---the least studied of the Pacific pidgins---is the most anglicized Melanesian pidgin.''\citep{nativization}
}\citep{nativization}。

事実、Pijin語についての研究資料として、語彙の面では非常に詳しく記された\cite{dictionary}があるが、包括的な文法書は未だに発行されておらず\citep{phonology}、文法は\cite{yumi}や\cite{eric}などの語学的な入門書から部分的に学ぶことしかできない。
トク・ピシンやビスラマ語がそれぞれの国で公用語の地位を与えられているのに対して、ソロモン諸島の公用語は未だに英語のみであるという政治的な地位の低さも、研究の少なさに拍車をかけている要因の1つといえるだろう。

本研究の目的は、近縁のピジン言語と比較しても未解明の部分が多く、近年の都市化や英語との言語接触によって激しい影響を受けていると言われる現代のPijin語文法を分析することであり、特にこの論文では、\textit{dat}と\textit{fo}という2つの単語の従属節標識としての用法の共時的な記述を目標とする。これら2つの単語は、既存の研究や語学資料には、本論で取り扱う従属節標識としての用法は示されていても非常に簡潔であるか、あるいは全く示されていない。それゆえ、これら2単語の従属節標識としての役割を示すことには一定の意味があるだろう。

本論では、まずはじめに本章の残りでPijin語の社会的状況や歴史的背景に対する簡単な説明をした上で、本研究で行ったPijin語聖書のコーパス分析や現地での調査について、その詳細を述べる。その後、第2章では\textit{dat}について、第3章では\textit{fo}について単語ごとにその分析や調査の結果、そして簡単な考察を述べる。第4章では、第3章までの結果を振り返り、Pijin語の文法研究全体についての示唆を述べるつもりである。

%筆者は未解明な部分の多く、比較的近年に発行されたPijin語新約聖書のコーパス分析を行った。その結果、本来は別の意味を持っている2つの単語---\textit{dat}、\textit{fo}---が、これまでの研究や辞書にはほとんど記されていない従属節標識としての用法を持っていることが明らかになった。その後、ソロモン諸島の首都であるホニアラに2週間滞在する機会を得たため、これらの用法や容認度について現地での調査を行った。
\section{Pijin語について}
\subsection{歴史と言語状況}
本論の目的は現代Pijin語のあくまで共時的な研究であるが、文法やその変化に関連があると思われるPijin語の社会言語学的な状況を整理するために、本節では簡単にPijin語の歴史を解説した上で現在の言語状況についても述べる。\footnote{
ここに示されたPijin語の歴史については、特に断りがなければ\cite{phonology}に基づいている。その他、この節では取り上げていないものの、Pijin語を中心としてメラネシア・ピジンの歴史に迫った詳細な研究として\cite{keesing}があり、1850年代以前に中央太平洋地域(カロライン諸島、ロツマ島、フィジー諸島、ギルバート諸島)で話されていたいわゆるジャーゴン言語から、早期メラネシア・ピジンがソロモン諸島のPijin語へと変化していく過程でのソロモン諸島の現地語(特にマライタ島のKwaio語)の影響についてまで詳細な記述と検討を行っている。}

太平洋のピジン言語は、その起源を19世紀前半にまで遡ることができる\citep{keesing}が、現在メラネシア・ピジンと呼ばれる3言語---トク・ピシン、ビスラマ語、そしてPijin語---の原型がはっきりと形作られるのは19世紀後半のオーストラリアである。1863年から1906年までの43年間、オーストラリアのクイーンズランドではメラネシア全域からプランテーションに従事する労働者が集められていたが、その中にソロモン諸島からの13,000人の労働者も含まれていた。そのプランテーションで話されていたのが、英語を主な語彙供給源としたまさに早期メラネシア・ピジンと呼ばれるピジン言語であり、彼らは自身の契約期間、あるいはプランテーション自体の終了後にその言語をソロモン諸島全域に持ち帰ることになったのである。

1893年、ソロモン諸島がイギリスにより併合されると、このピジン言語は現地の島民が植民地政府のイギリス人職員や諸島内の他民族と話す手段としての社会的な役割を担うことになった。20世紀に入り、諸島内でもプランテーションが開始すると、この言語はさらに多くの話者を抱えることになる。太平洋戦争においては、ソロモン諸島はガダルカナル島を含む各地が激戦地となったが、この時期においてもピジン言語はコミュニケーションの媒体として大いに役立った。2,000人程度の現地人がソロモン諸島労働者部隊として、680人がイギリス領ソロモン諸島保護国防衛軍として米軍側で戦った記録が残っているが、アメリカ軍の兵士は、この地域に入る前に基礎的なレベルのトク・ピシンを学んだ記録があり、島民側の証言では兵士と話すのに大半はピジン語を、部分的には英語を用いたという証言を残している。第二次世界大戦後の政治運動(Maasina Rule)でもPijin語はソロモン諸島内の異なる言語集団を結びつける大きな役割を果たし、独立後も都市化や学校教育\footnote{
一部の初等教育などの例外を除いて、学校で教科書や授業に用いられる言語は英語である一方で、より非公式な状況---登下校中や休憩時間の友達との会話---ではPijin語が用いられる。このため、学校に通う生徒は英語とPijin語の双方を覚えることが必須であり、さらにこれは後述する英語との激しいコードスイッチングの実例の1つといえるだろう。
}
の発展に伴って、Pijin語が使用される機会は劇的に増加していった。

ソロモン諸島は言語的に非常に多様であり、首都ホニアラでは64の土着語を話す集団がいるという\citep{nativization}。1つの土着語を話す集団はwantokと呼ばれ、島民は同じwantokの人の間ではその言語を、異なるwantokの間ではPijin語を用いるのがソロモン諸島の従来(あるいは非都市部)の言語習慣である。一方で、前述した通り多様な言語集団が併存し、wantokを跨いだ婚姻も増えているという首都ホニアラでは、1960年頃から日常的にPijin語が話されるようになり、さらに従来は他民族と話す際の手段でしかなかったPijin語を母語として話す人々が誕生するようになった。

Pijin語はソロモン諸島の事実上の\ruby[g]{共通言語}{リンガ・フランカ}ではあるが、公用語としての公的な地位は得ていない。これは文法の標準化・固定化をもたらさず、Pijin語の激しい変化や地域・話者による多様性を生む結果になっていると\cite{phonology}は指摘している。

さらに、Pijin語は話し言葉として頻繁に用いられるものの、書き言葉として用いられることはほとんどない\footnote{これは\cite{phonology}にも指摘されていることであるが、筆者の個人的な体験では、ホニアラ滞在中に書かれたPijin語を見たのは体感で全体の1~2割程度の街頭広告と、新聞の風刺漫画欄(通常の記事や広告、政府広報は国内で書かれたものも全て英語である)、そしてごく一部の注意書き(\textit{no smok}「喫煙禁止」など)であった。}。これは正書法の欠如などにも影響するが、本論の主題となる従属節標識節に対する既存の辞書や語学入門書での扱いの小ささにも密接に関連する。\cite{chafe}は書き言葉と比較した話し言葉の特徴として、従属節が導入される頻度が少なく、書き言葉では従属節で表される内容が独立した文あるいは並置されて話されることが多いということを述べていて、その例として本論の第2章で扱う英語\textit{that}節も登場する。すなわち、Pijin語は主として話し言葉として用いられているために、これら従属節標識が登場する頻度が相対的に少ないと考えられるのである。本論においては、比較的近年に発行された聖書をコーパス分析することにより、この話し言葉特有の従属節標識の少なさを克服していく。

\subsection{英語との関係}
これから従属節標識を取り扱っていく中で、形がよく似ている英語との比較を試みることがある。そのため、この節ではPijin語と英語の関係について簡単に述べる。

前述した通りPijin語はピジン・クレオール言語と呼ばれる言語のうちの1つであり、その語彙は主に(この言語が誕生した頃にオーストラリアや近隣地域で話されていたような)英語を供給源としており、特にPijin語は太平洋のピジン言語の中で最も英語からの影響を受けていると言われている\cite{nativization}。しかし、個々の単語などからは確かに簡単に英語が連想できるものの、次の章から示していく例文からも明らかなように、Pijin語と英語には一見して分かる隔たりがあり、特に文法の場合はそれが顕著である。\cite{malaitan}はPijin語と英語、そして現地で話される土着語の関係について、ソロモン諸島のマライタ島で話されている現地語とPijin語の文法的単語を比較した論文で次のように書いている。

\begin{quotation}
  英語のPijin語への影響は直ちに明らかである。Pijin語の辞書\citep{yumi}を見れば、そこに定義されている語の95\%が英語に由来することが分かる。地元のメラネシア語の影響も同様に明白で、広く行き渡っているが、一般的にはより捉えがたい。一般的に、英語の貢献は形式と主な意味のレベルでより大きく、一方、現地の言語の貢献は文法と派生的な意味のレベルでより大きくなっている。\citep{malaitan}\footnote{訳は発表者。原文: ``The influence of English on Pijin is immediately obvious. A scan through the Pijin dictionary (Simons and Young 1978) shows that 95 per cent of the words defined there have borrowed their form from English. However, the influence of local Melanesian languages is evident as well, equally as pervasive but generally more subtle. As a general rule the contribution of English has been more at the level of form and primary meaning, while the contribution of the local languages has been more at the level of grammar and extended meanings.''}
\end{quotation}

しかし、都市化や母語化に伴って、近年のPijin語は大きく変化している\footnote{以下この段落の内容は特に断りがない限り\cite{nativization}による。}。その中のいくつかは単なる単語の省略\footnote{\textit{long}/lo/や\textit{blong}/blo/、\textit{olketa}/ota/など。}ではあるが、その中には英語の影響をはっきりと受けていると感じられるものも少なく、それらの現象はPijin語の英語化(anglicization)として捉えられる。これら英語からの影響は特にラジオなどで顕著であり、ラジオで話される英語に近いPijin語は、そのいくらかの話者にとってPijin語にも英語にも感じられない言葉として認識される。

前節で述べたような社会的状況によって、都市部では英語とPijin語のコードスイッチングが極めて頻繁に行われる。そのため、同じ話者の発話でも、文によって英語とPijin語の間を行き来するということはよくあるし、場合によっては同じ文の中でも節によって英語であったりPijin語であったりする。

\section{調査の方法}\label{sec:howexamined}
次の章以降では、\textit{dat}, \textit{fo}という2つの単語について検討していくが、その際の例文や分布、容認度は主に2つのデータに依拠している。1つはPijin語訳の新約聖書であり、もう1つは筆者がソロモン諸島の首都ホニアラで行った容認度調査である。

Pijin語聖書は、\cite{solomontimes}によれば2008年に翻訳・出版されたもので、現地の翻訳支援機関であるSITAGも携わっている\footnote{現地で手に入れた本のクレジットでは翻訳者として``Wycliffe Bible Translators International''が示されている。Bible.isではテキストの提供者としてさらに``Bible Society of the South Pacific'', ``SIL Solomon Islands''の名前が挙げられている。}ことから、これまで書かれてきた文法書や辞書よりも新しいPijin語をある程度反映しているといえるだろう。本論では、この聖書のテキストに対するコーパス分析を行った。この際、元々の翻訳では新約部分と共に旧約部分も訳されているが、本論においては計算機的な分析の容易さを理由として、インターネット上に公開されている\footnote{Bible.is \url{https://live.bible.is/bible/PISWBT/}現地で出回っている2008年訳の聖書と同じ訳であることは部分的な比較からほぼ確実である。}新約部分のみを用いた。

また、次章以降で取り上げる現地での容認度調査は、2019年9月にソロモン諸島の首都ホニアラにおいて、高校生から60代までの9人の話者を対象に行われた。全員がホニアラに10年以上住んでいるが、出身がホニアラの人は3人だけで、Pijin語が母語だという人は1人だった。全員が学校教育を受けていて、Pijin語と英語双方の知識を持っていた。前節で述べたように、話者の居住地域や教育の有無、そして英語の日常的な使用の有無によってPijin語は大きく異なるため、今回の調査のインフォーマントが都市部に住んでおり、さらに英語で教育を受け知識を持っていることは留意しなければならない。ただし、話者の母語としている言語は全員が異なり、出身地もマライタ島\footnote{ホニアラに住んでいる人で最も多く出身者がいると言われる。}以外だけではないソロモン諸島全域にまたがっていたため、都市部で行われた調査としては比較的ソロモン諸島の多様な言語背景を反映できているとは言えるだろう。
