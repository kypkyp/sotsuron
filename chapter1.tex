\chapter{序論}

% アブスト
%
% ソロモン諸島の\ruby[g]{共通言語}{リンガ・フランカ}であるPijin語は、英語を主な語彙供給源とするメラネシアン・ピジンの1つであるが、包括的な文法書は未だに発行されておらず、公用語としての地位もないため、地域や時代による変化が激しいとされる。本論文では、現在のPijin語の文法に迫る1つの試みとして、\textit{dat}、\textit{fo}の従属節標識としての用法を、近年発行されたPijin語の聖書についてのコーパス分析と、ソロモン諸島の首都ホニアラでの容認度調査を元に示す。

英語を主な語彙供給源とするメラネシア・ピジンには、パプア・ニューギニアのトク・ピシンのTok Pisin、バヌアツのVislamaなどの変種が存在するが、ソロモン諸島の事実上の\ruby[g]{共通語}{リンガ・フランカ}であるPijin語は最も英語に近いと言われている。

\section{Pijin語について}

Pijin語はソロモン諸島の事実上の\ruby[g]{共通語}{リンガ・フランカ}である。

\section{先行研究}
