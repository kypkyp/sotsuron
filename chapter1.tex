\chapter{序論}

ソロモン諸島の\ruby[g]{共通言語}{リンガ・フランカ}であるPijin語
\footnote{
ソロモン諸島で話されているこの言語は、現地では単に\textit{pijin}あるいは\textit{tok pijin}と呼ばれる。英語の文献では種類としてのピジン言語を``pidgin language(s)''、この個別言語を``Pijin''あるいは``Solomon Islands Pijin(SIP)''と表記することが多いようである。本論においては、前者を「ピジン言語」、後者を「Pijin語」と呼んで区別することにする。}
は、名前の通りピジン・クレオール言語の1つであり、パプア・ニューギニアのトク・ピシン、バヌアツのビスラマ語と共にメラネシア・ピジンというグループを構成する
\footnote{
この3言語にトレス諸島クレオール(Torres Strait Creole)が加えられることもある(cf. \ref{fn:broken})。
}。
これらの中で、Pijin語は最も英語の影響を受けていると広く認められている一方、研究は3つの中で最も進んでいないとされる
\footnote{
Jourdanは次のように書いている:「ソロモン諸島のPijin語(SIP)---太平洋のピジンの中で最も研究されていない---は、最も英語化されたメラネシア・ピジンであるということは、太平洋のピジン・クレオール語学者たちの間にでさえ、広く伝わっている考えであるように思われる」
  ``It seems a widespread impression, even among Pacific pidgin and creole scholars, that Solomon Islands Pijin (SIP)---the least studied of the Pacific pidgins---is the most anglicized Melanesian pidgin.''\citep{nativization}
}。

事実、Pijin語についての研究資料として、語彙の面では非常に詳細な\cite{dictionary}があるが、包括的な文法書は未だに発行されておらず\citep{phonology}、文法は\cite{yumi}や\cite{eric}などの語学的な入門書から部分的に学ぶことしかできない。
トク・ピシンやビスラマ語がそれぞれの国で公用語の地位を与えられているのに対して、ソロモン諸島の公用語は未だに英語のみであるという政治的な地位の低さも、研究の少なさに拍車をかけている要因の1つといえるだろう。

本研究の目的は、近縁のピジン言語と比較しても未解明の部分が多く、近年の都市化や英語との言語接触によって激しい影響を受けていると言われる現代のPijin語文法を分析することであり、特にこの論文では、\textit{dat}と\textit{fo}という2つの単語の従属節標識としての用法の共時的な記述を目標とする。これら2つの単語の従属節標識としての用法は、既存の研究や語学資料では非常に簡潔な形でしか示されていないか、あるいは全く示されていない。それゆえ、本論でこれら2単語の従属節標識としての役割を示すことには一定の意味があるだろう。

本論では、まずはじめに本章の残りでPijin語の社会的状況や歴史的背景に対する簡単な説明をした上で、本研究で行ったPijin語聖書のコーパス分析や現地での調査について、その詳細を述べる。その後、第2章では\textit{dat}について、第3章では\textit{fo}について単語ごとにその分析や調査の結果、そして簡単な考察を述べる。第4章では、第3章までの結果を総括すると共に、今後のPijin語研究に向けた展望を述べる。

\section{Pijin語について}
\subsection{言語状況}
Pijin語は歴史的にはトク・ピシンやビスラマ語と起源の等しいメラネシア・ピジンの1言語であり、1893年のイギリスによるソロモン諸島併合以降、異なる言語を話す諸島民同士のコミュニケーションに用いる言語として社会的な地位を占めてきた
\footnote{
この節で示されるPijin語の言語状況は、特に断りがなければ\cite{phonology}に基づいている。その他、この節では取り上げていないものの、Pijin語を中心としてメラネシア・ピジンの歴史に迫った詳細な研究として\cite{keesing}があり、1850年代以前に中央太平洋地域(カロライン諸島、ロツマ島、フィジー諸島、ギルバート諸島)で話されていたいわゆるジャーゴン言語から、早期メラネシア・ピジンがソロモン諸島のPijin語へと変化していく過程でのソロモン諸島の現地語(特にマライタ島のKwaio語)の影響についてまで詳細な記述と検討を行っている。}。

ソロモン諸島は言語的に非常に多様であり、首都ホニアラには少なくとも64の土着語を話す集団がいるという\citep{nativization}。1つの土着語を話す集団はwantokと呼ばれ、同じwantokの人の間ではその言語を用いるが、異なるwantokの間でのコミュニケーションに利用されるのがPijin語である。そのため、Pijin語は基本的に第二言語であったが、前述した通り多様な言語集団が併存し、wantokを跨いだ婚姻も増えているという首都ホニアラでは、1960年頃から日常的にPijin語が話されるようになり、さらにPijin語の若い母語話者も生まれるようになっていった。

Pijin語はソロモン諸島の事実上の\ruby[g]{共通言語}{リンガ・フランカ}ではある一方で、公用語としての公的な地位は得ていない。これは発音や文法の標準化・固定化をもたらさず、Pijin語の激しい変化や地域・話者による多様性を生む結果になっていると\cite{phonology}は指摘している。

さらに、Pijin語は話し言葉として頻繁に用いられるものの、書き言葉として用いられることはほとんどない\footnote{
\cite{phonology}と、何人かのネイティブスピーカーの指摘による。筆者の個人的な体験では、ホニアラ滞在中に観察できた書かれたPijin語は体感で全体の1~2割程度の街頭広告と、新聞の風刺漫画欄(通常の記事や広告、政府広報は国内で書かれたものも全て英語である)、そしてごく一部の注意書き(\textit{no smok}「喫煙禁止」など)であった。ホニアラ市内の図書館の司書は、ほぼ全ての蔵書は英語であり、Pijin語で書かれたものは聖書だけであると述べた。}。これは正書法の欠如などにも影響するが、本論の主題となる従属節標識節に対する既存の辞書や語学入門書での扱いの小ささにも密接に関連する。\cite{chafe}は書き言葉では従属節などの文法的機能によって表される内容が、話し言葉では独立した文や並置によって表されるため、従属節の用いられる頻度が少ないということを話し言葉の特徴として指摘している。これは話し言葉としての使用が主であるPijin語において、従属節標識の記述が少ない原因の1つと考えられる。本論においては近年発行された聖書のコーパス分析によって、話し言葉だけでは補えない従属節標識の用例を収集した。

\subsection{英語との関係}
本論の目的はあくまでPijin語文法の共時的な記述であるが、形がよく似ている英語との比較を試みることがあるため、この節ではPijin語と英語の関係について簡単に述べる。

前述した通りPijin語はピジン・クレオール言語と呼ばれる言語のうちの1つであり、その語彙は主に(この言語が誕生した頃にオーストラリアや近隣地域で話されていたような)英語を供給源としており、特にPijin語は太平洋のピジン言語の中で最も英語からの影響を受けていると言われている\citep{nativization}。そのために、個々の単語などからは確かに簡単に英語が連想できるものの、次の章から示していく例文からも明らかなように、(特に伝統的な)Pijin語と英語には一見して分かる隔たりがあり、特に文法の場合はその違いが顕著である。\cite{malaitan}はPijin語と英語、そして現地で話される土着語の関係について、ソロモン諸島のマライタ島で話されている現地語とPijin語の文法的単語を比較した論文で次のように書いている。

\begin{quotation}
  英語のPijin語への影響は直ちに明らかである。Pijin語の辞書\citep{yumi}を見れば、そこに定義されている語の95\%が英語に由来することが分かる。地元のメラネシア語の影響も同様に明白で、広く行き渡っているが、一般的にはより捉えがたい。一般的に、英語の貢献は形式と主な意味のレベルでより大きく、一方、現地の言語の貢献は文法と派生的な意味のレベルでより大きくなっている。\citep{malaitan}\footnote{
  訳は著者。原文: ``The influence of English on Pijin is immediately obvious. A scan through the Pijin dictionary (Simons and Young 1978) shows that 95 per cent of the words defined there have borrowed their form from English. However, the influence of local Melanesian languages is evident as well, equally as pervasive but generally more subtle. As a general rule the contribution of English has been more at the level of form and primary meaning, while the contribution of the local languages has been more at the level of grammar and extended meanings.'' \citep{malaitan}}
\end{quotation}

しかし、都市化や母語化に伴って、近年のPijin語は大きく変化している\footnote{
以下この節の内容は特に断りがない限り\cite{nativization}による。}。その中のいくつかは単なる単語の省略\footnote{
\textit{long}/lo/や\textit{blong}/blo/、\textit{olketa}/ota/など。元々は綴り字通りの発音で、今でもそのように発音する人もいる。}ではあるが、その中には英語の影響をはっきりと受けていると感じられるものも少なくなく、それらの現象はPijin語の英語化(anglicization)として捉えられる。これら英語からの影響は特にラジオなどで顕著であり、ラジオで話される英語に近いPijin語は、それらの話者にとってPijin語にも英語にも感じられない言葉として認識される。そういった英語交じりの発話は伝統的なPijin語の話者から``\textit{Diskaen Pijin ia hemi rabis tumas}''「このPijin語は最悪だ」として糾弾されることもある。

前節で述べたような社会言語学的状況によって、ラジオ上の発話だけではなく日常的な会話においても、都市部では英語とPijin語のコードスイッチングが極めて頻繁に行われる。そのため、同じ話者の発話でも、文によって英語とPijin語の間を行き来するということはよくあるし、場合によっては同じ文の中でも節によって英語であったりPijin語であったりする。

これは、\cite{dictionary}もそこで取り上げる単語の選定基準についての注意書きで述べているように、何がPijin語で何がそうでないかという言語の定義すら難しくなる原因となる。次節で述べる調査やデータ収集の際にはなるべく全体がPijin語として安定しているものを選び、容認度調査の際にも伝統的なPijin語であっても容認できるかどうか尋ねるようにしたが、一部ラジオの録音などで収集した例文には、発音や文法において明らかに英語の影響を感じられるものも多い。このため、本論の分析結果の一部は伝統的な話者にとっては「Pijin語でない」と考えられるデータを含む可能性があることには注意しなければならない。

\section{調査の方法}\label{sec:howexamined}
次の章以降では、\textit{dat}, \textit{fo}という2つの単語について検討していくが、その際の例文や分布、容認度は主に2つのデータに依拠している。1つはPijin語訳の新約聖書であり、もう1つは筆者がソロモン諸島の首都ホニアラで行った容認度調査である。

Pijin語聖書は、\cite{solomontimes}によれば2008年に翻訳・出版されたもので、現地の翻訳支援機関であるSITAGも携わっている\footnote{
現地で手に入れた本のクレジットでは翻訳者として``Wycliffe Bible Translators International''が示されている。Bible.isではテキストの提供者としてさらに``Bible Society of the South Pacific'', ``SIL Solomon Islands''の名前が挙げられている。}ことから、これまで書かれてきた文法書や辞書よりも新しいPijin語をある程度反映しているといえるだろう。本論では、この聖書のテキストに対するコーパス分析を行った。この際、元々の翻訳では新約部分と共に旧約部分も訳されているが、本論においては計算機的な分析の容易さを理由として、インターネット上に公開されている\footnote{
Bible.is \url{https://live.bible.is/bible/PISWBT/}現地で出回っている2008年訳の聖書と同じ訳であることは部分的な比較からほぼ確実である。}新約部分のみを用いた。

また、次章以降で取り上げる現地での容認度調査は、2019年9月にソロモン諸島の首都ホニアラにおいて、高校生から60代までの9人の話者を対象に行われた。全員がホニアラに10年以上住んでいるが、出身がホニアラの人は3人だけで、Pijin語が母語だという人は1人だった。全員が学校教育を受けていて、Pijin語と英語双方の知識を持っていた。前節で述べたように、話者の居住地域や教育の有無、そして英語の日常的な使用の有無によってPijin語は大きく異なるため、今回の調査のインフォーマントが都市部に住んでおり、さらに英語で教育を受け知識を持っていることは留意しなければならない。ただし、話者の母語としている言語は全員が異なり、出身地もマライタ島\footnote{
ホニアラに住んでいる人で最も多く出身者がいると言われる。}以外だけではないソロモン諸島全域にまたがっていたため、都市部で行われた調査としては比較的ソロモン諸島の多様な言語背景を反映できているとは言えるだろう。また、一部の文章は問題となっている従属節標識以外の部分で不自然な点を指摘されたため、インフォーマントに協力してもらい、より自然な形に直してもらった。本論には最も多くの人に尋ねた形を書き残したが、一部のインフォーマントには細部の微妙に異なる例文を用いたことに注意する必要がある。
