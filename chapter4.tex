\chapter{結論}
ここまで第1章ではPijin語の概略や調査の手法について述べたあと、第2章では\textit{dat}の、第3章では\textit{fo}の従属節標識としての用法をみてきた。聖書の分析や容認度調査によって、従来の研究ではほとんど述べられてこなかったこれらの用法を確認できたことは本論の大きな成果である。

その一方、容認度調査で調査できた例文が4つと少なく、従属節標識の文中の機能を推測するにあたって十分ではなかったことは調査の反省点として挙げられる。また、序論で述べた通りPijin語は話者や地域による変種の差が大きいにも関わらず、調査の対象となったのは都市部で教育を受けた(英語の知識がある)話者に対してのみであったことも残念である。今後機会があれば、非都市部が大半を占めるソロモン諸島の各地で調査を行い、現在の多様なPijin語をさらに正確に反映する形で調査を行い、考察についてもなるべくPijin語の豊富な知識のある話者に確認する形で進めていきたい。

また、本論では話し言葉についての十分なデータが集められず、用例は聖書や著者による作文に頼らざるをえなかった。話し言葉において本論で取り上げた2つの単語が用いられている例として、最後にラジオで流されていた牧師の演説(\exn{1})やオンライン上で公開されている動画からの例文(\exn{2})を紹介する。

\begin{exe}
\ex\label{bokushi}
\gll Bat taem yu lukim gospel, ... yu mas save \underline{dat} end hemi kolsap nao \underline{fo} hemi kam...\\
but when 2SG look gospel ... 2SG must know \underline{that} end 3SG-PM close TOP \underline{for} 3SG-PM come\\
\glt `But when you look at the gospel, you must know that the end is coming soon.'
\ex\label{manguru1}
\gll Hem nao olsem main objective nao is \underline{fo} yumi stadim, documentim all mangrove blo yumi...\\
3SG TOP like main objective TOP (is) \underline{for} 2SG.PL study document all mangrove POSS 2SG.INC\\
\glt `Main objective for us is to study and document all mangroves that we share.'\citep[20分4秒]{manguru}
\end{exe}

両者とも内容自体はソロモン諸島に関するもので、英語からの翻訳であるとは考えにくいが、単語や構文で明らかに英語が用いられている部分があり、それゆえこれまでの章ではPijin語の例文として紹介することができなかった。序論で述べた通り、英語との頻繁なコードスイッチングは都市Pijin語の大きな特徴であり、時としてPijin語と英語の境界線を引くことすら困難を感じることがある。

このような不安定なPijin語の研究に際して、1つの固定化された形である聖書の分析は非常に役立った。聖書の発行は比較的最近(2008年)であり、この分析によって従来の話し言葉の分析だけでは難しかったPijin語の統語的な研究がより容易に実現できるようになるだろう。本論を通じてPijin語文法のいくらかを解明できただけではなく、そのような研究の1つの形を示すことができたならば幸いである。
