\chapter{fo}
この章では、従来は名詞の与格を示したり、動詞とともに不定詞を作ったりする機能が示されていた前置詞の\textit{fo}に、主語と目的語を従えた従属節を作る用例があることを示す。

\section{従来の意味}
\textit{fo}はメラネシア・ピジンの中でもPijin語とトレス海峡クレオール\footnote{\label{fn:broken}
オーストラリアのトレス海峡諸島で話される個別言語で、\cite{prepositions}では基本的にBrokenと表記されている。メラネシア・ピジンの枠組みには1章で述べた3言語のみが示されることが多いが、\cite{prepositions}は\cite{keesing}の提案に従い、この言語をメラネシア・ピジンのグループに加えている。}\footnote{トレス海峡クレオールにおいては、この前置詞は\textit{po}と表記される\citep{prepositions}。}にしか見られない前置詞で、名詞の前では与格として利益の享受などの関係を示す用法があり、動詞の前では英語の\textit{to}不定詞のような用法で用いることができ、この構文はしばしば目的を示す\citep{prepositions}。以下は\cite{prepositions}中に挙げられたのと同一の例文であるが、(\exn{1})は名詞の前での与格標識としての用法、(\exn{2})(\exn{3})は不定詞としての\textit{fo}の使用例となっている。

\begin{exe}
\ex
\gll Mitufala tekem kam samfala sugaken \underline{fo} iu.\\
1DU.EXC take DIR some sugarcane \underline{DAT} 2SG\\
\glt `We brought you some sugarcane.' \cite[44]{rr2}

\ex
\gll An hemi baebae baem samfala tul \underline{fo} waka long hem.\\
and 3SG-PM FUT buy some tool \underline{to} work LOC 3SG\\
\glt `And he'll buy some tools to work with.' \cite[270]{todd}

\ex
\gll Ating iu kam \underline{fo} spoelem mifala ia!\\
probably 2SG come \underline{to} destroy 1PL.EXC STATM\\
\glt `Probably you've come to destroy us!' (MRK 1:24)
\end{exe}

与格としての用法である(\exn{-2})では、\textit{fo}は二人称単数代名詞\textit{iu}の直前に来て、動詞句\textit{tekem kam}「取ってくる」の利益の享受者を示している。目的を示す用法として、(\exn{-1})ではその道具の目的である\textit{waka long hem}「それによって働く」を示す句を作って直前の名詞句\textit{samfala tul}「いくらかの道具」を形容詞的に修飾している。(\exn{0})では動詞\textit{kam}「来る」の直後に来るが、ここでは副詞的に\textit{spoelem mifala}「私たちを滅ぼす」という目的を補っているというように解釈できる。

\section{従属節標識として}
\subsection{聖書の用法}
聖書には、\textit{fo}が文全体に対して従属節を作っているように解釈できる用例が多く存在する。
(\exn{1})では\textit{olketa}と\textit{falom}、(\exn{2})では\textit{yu}と\textit{stap}というように、明らかに節の主語と述語として解釈できる語が\textit{fo}に続いていることが分かる。

\begin{exe}
\ex
\gll Yu mas tokstrong long olketa \underline{fo} olketa mas falom.\\
2SG must speak.strongly to 3PL \underline{to} 3PL must follow\\
\glt `You must speak to them strongly enough to follow you.' (1T 6:3)
\ex
\gll Hemi tambu long Lo \underline{fo} yu stap wetem disfala woman ya.\\
3SG-PM prohibitted in Law \underline{for} 2SG be with this woman DEIC\\
\glt `It is prohibitted in Law for you to be with this woman.' (MAT 14:4)
\end{exe}

これらは前置詞や不定詞句に留まらない従属節標識としての\textit{fo}の用法があることを示している。

\subsection{現地調査}
この\textit{fo}の従属節標識としての容認度について、\label{sec:howexamined}節で示した方法で調査を行った。調査には筆者が作った(\exn{1})(\exn{2})を用いた。

\begin{exe}
\ex
\gll Mi laik \underline{fo} yu mekem tok blo yu tru.\\
1SG want \underline{to} 2SG make telling POSS 2SG true\\
\glt `I want you to talk truth.'
\ex\label{ex:tambufo}
\gll Hemi tambu \underline{fo} mifala go insaet long disfala ples.\\
3SG-PM prohibitted \underline{to} 1PL.EXC go inside LOC this place\\
\glt `It is prohibitted for us to enter this place.'
\end{exe}

結果、こちらの文も全て適格であるとみなされた。インフォーマントに自然な言い換え表現を依頼したところ、この前置詞句の後置を問題にしたものはほぼ無かったが、50代のインフォーマント\footnote{\ref{fn:baelelea}と同一の話者}は次のようなものを提案した。

\begin{exe}
\exi{(\exn{0}$'$)} Hemi tambu \underline{fo} mifala \underline{fo} go insaet long disfala ples.
\end{exe}

(\exn{0})との意味の違いはないが、こちらのほうがより「強い」印象を受けるという。\textit{fo}$+$単語$+$\textit{fo}の組み合わせは聖書コーパス中106件が見つかる。元々2つの前置詞句からなっていた表現が、後ろの前置詞が省略されて節のようになったのかもしれない。ただし、それ以外の話者からこの置き換えが提案されることはなかった\footnote{7人で、Pijin語を母語とする高校生の話者も含む。(\exn{0})も(\exn{0}$'$)も同様に適格な文章であると判断された。}。
\section{考察}
\subsection{文章中の役割}
\ref{sec:vsthat}節に述べた通り、英語での(a)主語, (c)主格補語, (e)形容詞補語 としての\textit{that}節の役割はPijin語では\textit{fo}が代行する。(\ref{ex:tambufo})の\textit{fo}節は(a)の用例であり、文主語に当たる部分\textit{mifala go insaet long disfala ples}「私達がこの場所に入ること」が\textit{fo}節で現されているのが分かる。さらに(\ref{ex:tambufo})では文主語の位置に三人称代名詞\textit{hem}が置かれ\footnote{述語標識\textit{i}と結びつくため、例文上は\textit{hemi}という形を取る。}、\textit{fo}節は述語の後ろに置かれているが、このいわゆる形式主語構文は英語で主語に\textit{that}節や不定詞節が来る際にも典型的な現象である\citep[1049, 1391--1392]{english}。これは文構造のレベルでの英語からの影響を示唆する現象といえるかもしれない。

冗長性
