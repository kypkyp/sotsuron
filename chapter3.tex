\chapter{fo}
この章では、従来は名詞の与格を示したり、動詞とともに不定詞を作ったりする機能が示されていた前置詞の\textit{fo}に、主語と目的語を従えた従属節を作る用例があることを示す。

\section{従来の意味}\label{sec:originalfo}
\textit{fo}はメラネシア・ピジンの中でもPijin語とトレス海峡クレオール\footnote{\label{fn:broken}
オーストラリアのトレス海峡諸島で話される個別言語で、\cite{prepositions}では基本的にBrokenと表記されている。メラネシア・ピジンの枠組みには1章で述べた3言語のみが示されることが多いが、\cite{prepositions}は\cite{keesing}の提案に従い、この言語をメラネシア・ピジンのグループに加えている。}\footnote{トレス海峡クレオールにおいては、この前置詞は\textit{po}と表記される\citep{prepositions}。}にしか見られない前置詞で、名詞の前では与格として利益の享受などの関係を示す用法があるほか、動詞の前では英語の\textit{to}不定詞のような用法で用いることができ、この構文はしばしば目的を示す\citep{prepositions}。以下は\cite{prepositions}中に挙げられたのと同一の例文であるが、(\exn{1})は名詞の前での与格標識としての用法、(\exn{2})は不定詞としての\textit{fo}の使用例となっている。

\begin{exe}
\ex
\gll Mitufala tekem kam samfala sugaken \underline{fo} iu.\\
1DU.EXC take DIR some sugarcane \underline{DAT} 2SG\\
\glt ``We brought you some sugarcane." \cite[44]{rr2}

\ex
\gll An hemi baebae baem samfala tul \underline{fo} waka long hem.\\
and 3SG-PM FUT buy some tool \underline{to} work LOC 3SG\\
\glt ``And he'll buy some tools to work with." \cite[270]{todd}
\end{exe}

与格としての用法である(\exn{-1})では、\textit{fo}は二人称単数代名詞\textit{iu}の直前に来て、動詞句\textit{tekem kam}「取ってくる」の利益の享受者を示している。目的を示す用法として、(\exn{0})ではその道具の目的である\textit{waka long hem}「それによって働く」を示す句を作って直前の名詞句\textit{samfala tul}「いくらかの道具」を形容詞的に修飾している。

\section{従属節標識として}
\subsection{聖書の用法}
聖書には、\textit{fo}が文全体に対して従属節を作っているように解釈できる用例が多く存在する。
(\exn{1})では\textit{olketa}と\textit{falom}、(\exn{2})では\textit{yu}と\textit{stap}というように、明らかに節の主語と述語として解釈できる語が\textit{fo}に続いていることが分かる。

\begin{exe}
\ex\label{ex:purposefo}
\gll Yu mas tokstrong long olketa \underline{fo} olketa mas falom.\\
2SG must speak.strongly to 3PL \underline{to} 3PL must follow\\
\glt ``You must speak to them strongly enough to follow you." (1T 6:3)
\ex\label{ex:tambufo}
\gll Hemi tambu long Lo \underline{fo} yu stap wetem disfala woman ya.\\
3SG-PM prohibitted in Law \underline{for} 2SG be with this woman DEIC\\
\glt ``It is prohibitted in Law for you to be with this woman." (MAT 14:4)
\end{exe}

単純な前置詞句との統計的な区別が難しいため、正確に用例を数え上げることは難しい。一方で、述語標識の含まれる代名詞との単語連続\textit{fo hemi}の形が100例存在することからも、従属節標識としての\textit{fo}の用法が聖書内に頻繁に現れることは確実である、

\subsection{現地調査}
この\textit{fo}の従属節標識としての容認度について、\label{sec:howexamined}節で示した方法で調査を行った。調査には筆者が作った(\exn{1})(\exn{2})を用いた。

\begin{exe}
\ex\label{ex:laikfo}
\gll Mi laik \underline{fo} yu mekem tok blo yu tru.\\
1SG want \underline{to} 2SG make telling POSS 2SG true\\
\glt ``I want you to talk truth."
\ex
\gll Hemi tambu \underline{fo} mifala go insaet long disfala ples.\\
3SG-PM prohibitted \underline{to} 1PL.EXC go inside LOC this place\\
\glt ``It is prohibitted for us to enter this place."
\end{exe}

結果、こちらの文も全て適格であるとみなされた。インフォーマントに自然な言い換え表現を依頼したところ、この前置詞句の後置を問題にしたものはほぼ無かったが、50代のインフォーマント\footnote{\ref{fn:baelelea}と同一の話者}は次のようなものを提案した。

\begin{exe}
\exi{(\exn{0}$'$)} Hemi tambu \underline{fo} mifala \underline{fo} go insaet long disfala ples.
\end{exe}

(\exn{0})との意味の違いはないが、こちらのほうがより「強い」印象を受けるという。\textit{fo}$+$単語$+$\textit{fo}の組み合わせは聖書コーパス中106件が見つかる。元々2つの前置詞句からなっていた表現が、後ろの前置詞が省略されて節のようになったのかもしれない。ただし、それ以外の話者からこの置き換えが提案されることはなかった\footnote{7人で、Pijin語を母語とする高校生の話者も含む。(\exn{0})も(\exn{0}$'$)も同様に適格な文章であると判断された。}。
\section{考察}
これまで述べた通り、\textit{fo}は元から不定詞としての機能があり、\textit{fo}節は不定詞とその意味上の主語を示す前置詞句が合わさった形として解釈することができる。筆者の収集した用例においても、その文章中の機能は\textit{fo}不定詞単独の場合と同じに思われた。

\cite{english}は\textit{to}不定詞の用例として(a)主語 (b)直接目的語 (c)主格補語 (d)同格 (e)形容詞補語 の5つを挙げていて、筆者の調査では聖書内に(a)(b)(e)の用例を見つけることができた。(\ref{ex:tambufo})は\textit{fo}節が(a)の役割を、(\ref{ex:laikfo})は(b)または(e)の役割を担っている例である\footnote{Pijin語には状態動詞とも形容詞とも解釈できる多数の単語が存在し\cite[xvi]{dictionary}、大まかには「欲しい」を意味する単語\textit{laik}もその1例と考えられる。このような場合、状態動詞の直接目的語と形容詞補語を厳密に区別することは難しい。
}。

(\ref{ex:tambufo})では、文主語に当たる部分\textit{yu stap wetem disfala woman ya}「あなたがこの女と共にいること」が\textit{fo}節で現されているが、さらに(\ref{ex:tambufo})では\textit{fo}節は述語\textit{tambu}の後ろに置かれ、文主語の位置には三人称代名詞\textit{hem}が置かれている\footnote{述語標識\textit{i}と結びつくため、例文上は\textit{hemi}という形を取る。}。このいわゆる形式主語構文は英語で主語に\textit{that}節や不定詞節が来る際にも典型的な現象であり\citep[1049, 1391--1392]{english}、Pijin語における同様の構文の現れは、構文レベルでの英語からの影響を示している可能性がある。

また、\ref{sec:originalfo}節に示されたように、不定詞としての\textit{fo}には目的を示す用法が元々存在するが、節標識としても目的を示す節を形成できる。(\ref{ex:purposefo})はその例であり、\textit{olketa mas falom}「彼らが従う」という文全体の目的が\textit{fo}節によって示されている。
