\chapter{fo}
この章では、従来は名詞の与格を示したり、動詞とともに不定詞を作ったりする機能が示されていた前置詞の\textit{fo}に、主語と目的語を従えた従属節を作る用例があることを示す。

\section{既存の意味}
\textit{fo}はメラネシア・ピジンの中でもPijin語とトレス海峡クレオール\footnote{\label{fn:broken}
オーストラリアのトレス海峡諸島で話される個別言語で、\cite{prepositions}では基本的にBrokenと表記されている。メラネシア・ピジンの枠組みには1章で述べた3言語のみが示されることが多いが、\cite{prepositions}は\cite{keesing}の提案に従い、この言語をメラネシア・ピジンのグループに加えている。}\footnote{トレス海峡クレオールにおいては、この前置詞は\textit{po}と表記される\citep{prepositions}。}にしか見られない前置詞で、名詞の前では与格として利益の享受などの関係を示す用法があり、動詞の前では英語の\textit{to}不定詞のような用法で用いることができ、この構文はしばしば目的を示す\citep{prepositions}。以下は\cite{prepositions}中に挙げられたのと同一の例文であるが、(\exn{1})は名詞の前での与格標識としての用法、(\exn{2})(\exn{3})は不定詞としての\textit{fo}の使用例となっている。

\begin{exe}
\ex
\gll Mitufala tekem kam samfala sugaken \underline{fo} iu.\\
1DU.EXC take DIR some sugarcane \underline{DAT} 2SG\\
\glt `We brought you some sugarcane.' \cite[44]{rr2}

\ex
\gll An hemi baebae baem samfala tul \underline{fo} waka long hem.\\
and 3SG-PM FUT buy some tool \underline{to} work LOC 3SG\\
\glt `And he'll buy some tools to work with.' \cite[270]{todd}

\ex
\gll Ating iu kam \underline{fo} spoelem mifala ia!\\
probably 2SG come \underline{to} destroy 1PL.EXC STATM\\
\glt `Probably you've come to destroy us!' (MRK 1:24)
\end{exe}

与格としての用法である(\exn{-2})では、\textit{fo}は二人称単数代名詞\textit{iu}の直前に来て、動詞句\textit{tekem kam}「取ってくる」の利益の享受者を示している。目的を示す用法として、(\exn{-1})ではその道具の目的である\textit{waka long hem}「それによって働く」を示す句を作って直前の名詞句\textit{samfala tul}「いくらかの道具」を形容詞的に修飾している。(\exn{0})では動詞\textit{kam}「来る」の直後に来るが、ここでは副詞的に\textit{spoelem mifala}「私たちを滅ぼす」という目的を補っているというように解釈できる。

\section{従属節標識として}
\subsection{聖書の用法}
聖書には、\textit{fo}がその直後に主語と述語として解釈できる単語を伴い、文全体に対して従属節を作っているというように解釈できる用例が多く存在する。

\begin{exe}
\ex
\gll Yu mas tokstrong long olketa \underline{fo} olketa mas falom.\\
2SG must speak.strongly to 3PL \underline{to} 3PL must follow\\
\glt `You must speak to them strongly enough to follow you.' (1T 6:3)
\ex
\gll Hemi tambu long Lo \underline{fo} yu stap wetem disfala woman ya.\\
3SG-PM prohibitted in Law \underline{for} 2SG be with this woman DEIC\\
\glt `It is prohibitted in Law for you to be with this woman.' (MAT 14:4)
\end{exe}

\subsection{現地調査}

\section{考察}
