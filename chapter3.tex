\chapter{fo}
\section{既存の意味}
\textit{fo}はメラネシア・ピジンの中でもPijin語とトレス海峡クレオール\footnote{\label{fn:broken}
オーストラリアのトレス海峡諸島で話される個別言語で、\cite{prepositions}では基本的にBrokenと表記されている。メラネシア・ピジンの枠組みには1章で述べた3言語のみが示されることが多いが、\cite{prepositions}は\cite{keesing}の提案に従い、この言語をメラネシア・ピジンのグループに加えている。}\footnote{トレス海峡クレオールにおいては、この前置詞は\textit{po}と表記される\citep{prepositions}。}にしか見られない前置詞で、名詞の前では与格として利益の享受などの関係を示す用法があり、動詞の前では英語の\textit{to}不定詞のような用法で用いることができ、この構文はしばしば目的を示す\citep{prepositions}。以下は\cite{prepositions}中に挙げられたのと同一の例文である\footnote{(\exn{1}), (\exn{2}), (\exn{3})共にグロスは著者による。}が、(\exn{1})は名詞の前での与格標識としての用法、(\exn{2})(\exn{3})は不定詞としての\textit{fo}の使用例となっている。

\begin{exe}
\ex
\gll Mitufala tekem kam samfala sugaken \underline{fo} iu.\\
1DU.EXC take DIR some sugarcane \underline{DAT} 2SG\\
\glt `We brought you some sugarcane.' \cite[44]{rr2}

\ex
\gll An hemi baebae baem samfala tul \underline{fo} waka long hem.\\
and 3SG-PM FUT buy some tool \underline{to} work LOC 3SG\\
\glt `And he'll buy some tools to work with.' \cite[270]{todd}

\ex
\gll Ating iu kam \underline{fo} spoelem mifala ia!\\
probably 2SG come \underline{to} destroy 1PL.EXC STATM\\
\glt `Probably you've come to destroy us!' (MRK 1:24)
\end{exe}

与格としての用法である(\exn{-2})では、\textit{fo}は二人称単数代名詞\textit{iu}の直前に来て、動詞句\textit{tekem kam}「取ってくる」の利益の享受者を示している。目的を示す用法として、(\exn{-1})ではその道具の目的である\textit{waka long hem}「それによって働く」を示す句を作って直前の名詞句\textit{samfala tul}「いくらかの道具」を形容詞的に修飾している。(\exn{0})では動詞\textit{kam}「来る」の直後に来るが、ここでは副詞的に\textit{spoelem mifala}「私たちを滅ぼす」という目的を補っているというように解釈できる。

\section{聖書中の用法}
\section{現地調査}
