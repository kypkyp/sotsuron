\chapter{dat}

本章では、従来の辞書では指示詞(を形成する語幹)として挙げられている\textit{dat}に、名詞節を作る従属節標識としての用法があることを示す。さらに、意味的に似た場面で用いられる\textit{olsem}との分布の違いについても述べられる。その結果、\textit{dat}は特定の動詞の後ろに来て、文中の働きとしては直接目的語にあたる名詞節を形成する用法しか持たないことが示される。

\section{既存の意味}

\cite{dictionary}には\textit{dat}単体の形では語彙として挙げられていない一方で、\textit{datfala}「その」「それ(具体的)」、\textit{datwan}「それ」という形が掲載されている。(\exn{1})は\cite{dictionary}の\textit{datfala}の項で挙げられている用例であり\footnote{以下、特にことわりが無ければグロスは筆者。グロスで用いる略号は基本的に\cite{prepositions}に従ったが、いくつかの必要な略号は補った: 1/2/3=1st/2nd/3rd person, SG=singular, DU=dual, PL=plural, INC=inclusive, EXC=exclusive, DEIC=deictic, PM=predicate marker, FUT=future, PERF=perfective, NEG=negative, DAT=dative, LOC=locative, POSS=possessive, TOP=topic}、前置の修飾語として用いられているのがわかる。

\begin{exe}
  \ex
  \gll Datfala tri ia, hem long lan blong mi ia.\\
  that tree DET 3SG LOC land POSS 1SG DET\\
  \glt `That tree is on my land.'
\end{exe}

\textit{-fala}は主として形容詞接辞、\textit{-wan}は形容詞や状態動詞を名詞に変える働きを持つ接辞である\footnote{この定義は\cite{syntax}に従った。\textit{-fala}の形が定着した単語においては、直後に見る\textit{datfala}の名詞用法、三人称双数代名詞としての\textit{tufala}、一人称複数代名詞としての\textit{mifala}のように、共時的には必ずしも形容詞接辞として解釈できない意味を持つ単語もある。}一方で、徐々に多くの話者から省略されるようになってきている\citep{syntax}。
そのため、\textit{dat}単独での形でも、「その」「それ」という代名詞や限定詞としての意味を持つと推測され、事実インフォーマントへの調査でも限定詞としての用法が\textit{dat}の第一の語義として挙げられることが多かった。

\section{聖書中の用法}
\section{現地調査}
